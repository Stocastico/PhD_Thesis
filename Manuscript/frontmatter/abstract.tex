% !TeX spellcheck = en_GB

% Thesis Abstract -----------------------------------------------------


%\begin{abstractslong}    %uncommenting this line, gives a different abstract heading


\begin{abstracts}        %this creates the heading for the abstract page
%\selectlanguage{english}
\addcontentsline{toc}{chapter}{Abstract}
%Put your abstract or summary here.

In recent years, thanks to the availability of powerful mobile devices, the release of specialized libraries and the shrinking cost of head mounted displays, the number of Augmented Reality applications has grown exponentially. Initially deployed for manufacturing applications, Augmented Reality is now used in many other fields, among others gaming and entertainment, cultural heritage, customer engagement, and education. 

Focusing on education, Augmented Reality has a huge potential to revolutionise the sector. Even though learning applications based on Augmented Reality already exist, their creation is complex and they are mainly intended for individual usage. The lack of tools for synchronizing Augmented Reality experiences across several users, the issues related to the adaptation of the augmented content on different devices, the difficulty of incorporating applications into the learning management systems used in schools have, so far, limited the adoption of Augmented Reality in classroom settings. 

To address these issues, this research presents \arch{} (Collaborative Learning Environment for Augmented Reality), a novel architecture that enables the development of interoperable and collaborative Augmented Reality applications. The architecture has been designed taking into accounts both technical and educational aspects. In fact, several teachers from Basque primary and secondary schools helped in the definition of the architecture requirements. \arch{} is a modular architecture which also provides teachers tools for analysing the data about usage of Augmented Reality applications as well as the students results. Furthermore, since \arch{} relies on open standards, it can in principle be easily integrated with the schools data collection systems.

To evaluate the architecture design, a multiplatform, collaborative Augmented Reality application has been developed and tested in three educational institutions. The evaluation included collecting survey responses from the students who participated in the trials, interviewing their teachers and performing a quantitative analysis of the data collected through the application.

This work has been done in the context of the ARETE project - Augmented Reality Interactive Educational System, a H2020 EU-funded research project which aimed to investigate and define novel tools for collaborative Augmented Reality applications.


\end{abstracts}

%\begin{resumen}        %this creates the heading for the abstract page
%%\selectlanguage{spanish}
%Añadir aquí un resumen.
%
%
%\end{resumen}


%\begin{laburpena}        %this creates the heading for the abstract page
%\selectlanguage{basque}
% Jarri zure laburpena hemen.
%Laburpena Euskaraz.

%\end{laburpena}

%\end{abstractlongs}


% ---------------------------------------------------------------------- 
