\chapter{Teacher survey} \label{append:teachsur}
This is the survey that was presented to the teachers and whose results helped defining the requirements of the architecture presented in Chapter \ref{chap:arch}. The survey was created as a web-based form and, depending on how a teacher answered, not all the questions were presented. The survey included also some background information as well as images. Furthermore, some of the questions only admitted answers from a predefined set. We omitted these redundant information for brevity, but the interested reader can find the full version of the survey online\footnote{\url{https://anon.to/oBIuF6}}.

%\renewcommand{\arraystretch}{1.1}
\begin{longtable}{p{0.05\textwidth}>{\arraybackslash}p{0.92\textwidth}}
\toprule
 & Question\\
\midrule
\endhead
    \textbf{Q1} & What is the main subject you teach? \\
    \textbf{Q2} & How many years of experience do you have? \\
    \textbf{Q3} & What is the average number of students per class in your school? \\
    \textbf{Q4} & What is the educational level of your classes? \\
    \textbf{Q5} & Are you a dynamizer in your school? \\
    \textbf{Q6} & How many smartphones are available in your schools? \\
    \textbf{Q7} & How many tablets are available in your schools? \\
    \textbf{Q8} & How many desktop PCs are available in your schools? \\
    \textbf{Q9} & How many laptops are available in your schools? \\
    \textbf{Q10} & If other devices are available, please specify the type of device and the number. \\
    \textbf{Q11} & Have you ever used technological tools to facilitate student learning,
beyond office tools or video calls? These tools can be mobile applications, web
applications, 3D visualizations, simulations... \\
    \textbf{Q12} & What tools have you used? \\
    \textbf{Q13} & What aspects would you highlight of the tools you have used? \\
    \textbf{Q14} & Which of the following aspects are present in the tools you have used? \\
    \textbf{Q15} & Have you used augmented reality during your teaching years? \\
    \textbf{Q16} & Do you think that augmented reality applications could facilitate the learning of your students? \\
    \textbf{Q17} & Would you like to use augmented reality more often during your classes? \\
    \textbf{Q18} & What do you need to use (or increase the use of) augmented reality in your teaching? \\
    \textbf{Q19} & What kind of devices have you used to teach with augmented reality? \\
    \textbf{Q20} & How would you evaluate your level of satisfaction using augmented reality in your teaching? \\
    \textbf{Q21} & How should AR apps change to improve your satisfaction when using them? \\
    \textbf{Q22} & How comfortable do you feel using augmented reality in your teaching? \\
    \textbf{Q23} & Do you believe that using AR apps in class has favored students learning? \\
    \textbf{Q24} & What kind of applications (not necessarily augmented reality)
would you like to use in class? \\
    \textbf{Q25} & How would your students use this app? \\
    \textbf{Q26} & Do you think that augmented reality can be an added value in teaching? \\
    \textbf{Q27} & What do you think are the advantages of augmented reality when using it at school? \\
    \textbf{Q28} & And what are its drawbacks? \\
    \textbf{Q29} & Do you think that technology (not necessarily augmented reality) can help teachers measure student learning and the ability to retain the subjects studied? \\
    \textbf{Q30} & What features are you most interested in in an augmented reality application? \\
    \textbf{Q31} & If you had an application to create augmented reality educational content, would you use it to create your own app? \\
    \textbf{Q32} & What kind of educational content would you create with that application? \\
    \textbf{Q33} & In the European project ARETE, we are developing software that allows us to easily create collaborative augmented reality applications. How do you think you could use this technology
in your work? \\
    \textbf{Q34} & We are also developing artificial intelligence applications to facilitate
the work of teachers. How do you think you could use artificial intelligence in your work? \\
\bottomrule
\end{longtable}
\label{tab:teachersurv}



\chapter{Student questionnaire} \label{append:stuqu}
The following is the questionnaire that students were asked to fill after using \appname{}, the application described in Chapter \ref{chap:eval}. Students were asked to fill a 20-item subjective questionnaire, using a Likert scale from 1 to 5 to assess their agreement with each sentence. The items in the questionnaire belong to four different clusters, depending on the aspect to evaluate (collaborative aspects, app usability, functionality, interest as an educational tool). Questions Q12 and Q15 were filled only by students who used the mobile applications, as the ones using the web interface could not receive suggestions but only provide them. Q18 and Q20 were framed slightly differently for users on a mobile device or using a PC: for the former group, the question referred to the usage of AR, while for the latter it was about the inclusion of 3D elements in the application. 


\begin{table*}[ht]\centering
\begin{tabular}{p{0.05\textwidth}>{\arraybackslash}p{0.92\textwidth}}
%\begin{longtable}{p{0.05\textwidth}>{\arraybackslash}p{0.92\textwidth}}
\toprule
 & Question\\
\midrule
    \textbf{Q1} & I think that I would like to use the application frequently. \\
    \textbf{Q2} & I found the application to be simple. \\
    \textbf{Q3} &I thought the application was easy to use. \\
    \textbf{Q4} & I think that I could use the application without the support of a technical person. \\
    \textbf{Q5} & I found the various functions in the application were well integrated. \\
    \textbf{Q6} & I would imagine that most people would learn to use the application very quickly. \\
    \textbf{Q7} & I found the application very intuitive. \\
    \textbf{Q8} & I felt very confident using the application. \\
    \textbf{Q9} & I could use the application without having to learn anything new. \\
    \textbf{Q10} & I would like to use the application during a test. \\
    \textbf{Q11} & Being able to provide suggestions made me feel more involved. \\
    \textbf{Q12} & Receiving suggestions made me more confident when answering a question. \\
    \textbf{Q13} & At all times I have been able to understand what the person who had to respond to the exercise was doing. \\
    \textbf{Q14} & I find it more interesting to solve the exercises through the application than through a web page or in writing. \\
    \textbf{Q15} & Suggestions from my classmates have helped me when answering the exercise. \\
    \textbf{Q16} & The device used  has allowed me to use the application easily. \\
    \textbf{Q17} & I would like to use the application to learn new concepts. \\
    \textbf{Q18} & Being able to use augmented reality/ 3D elements makes the application more entertaining. \\
    \textbf{Q19} & There are several ways to collaborate with my classmates through the application. \\
    \textbf{Q20} & Thanks to augmented reality / 3D elements I have felt immersed in the learning activity. \\
\bottomrule
%\end{longtable}
\label{tab:studentquestionnaire}
\end{tabular}
\end{table*}

\chapter{Résumé}

Stefano Masneri received the B.Sc. degree in information technology and the M.Sc. degree in telecommunications from the Università degli Studi di Brescia, Italy, in 2005 and 2008, respectively. He has worked at several research institutions (Vicomtech, Max Planck Institute for brain research, Fraunhofer HHI for telecommunications, CNIT) in Spain, Germany and Italy. There he worked on many national and European research projects in areas such as computer vision, signal processing, data analysis and augmented reality. He is now working at NTT DATA as a technical manager, overseeing several projects related to the implementation of generative artificial intelligence in the banking and energy sector.
