% Thesis Acknowledgements ------------------------------------------------


% Opening of the acknowledgements

%Sort version
%this creates the heading for the acknowlegments
%\begin{acknowledgements}
%Long version
%uncommenting this line, gives a different acknowledgements heading
\begin{acknowledgementslong} 
\addcontentsline{toc}{chapter}{Acknowledgements}

While the cover lists only my name as author, this work would not have been possible without the contributions of many people who helped a lot over the last four years.

First and foremost, many thanks to Ana Arruarte and Mikel Zorrilla, my Ph.D. supervisors, for their support, guidance and involvement. Mikel Larrañaga did not have an official role, but he has been acting since the start as the third supervisor and has greatly helped me.

I would also like to thank my former Vicomtech colleagues, who contributed to this work even after I left the company and without any obligation to do so. Ana Domínguez has been invaluable in this sense and my Ph.D. would not have been half as good without her contributions and support. She mentioned once that she hoped to help me as much as I helped her during her Ph.D, but she went far beyond that. Iñigo Tamayo and Guillermo Pacho developed most of the code and helped greatly in the validation of the architecture. Without their help \arch{} and \appname{} would probably exist only in my head. The rest of the Vicomtech team helped and supported a lot, especially Roberto Viola, Dorleta García, Álvaro Gabilondo and Miguel Sanz. 

The Ikastolen Elkartea association played a key role by helping me get in touch with many teachers. The teachers filled in the survey and their answer enabled us to define the requirements of the architecture. Furthermore, they provided important feedback in many phases of the development process. Later, Nahia Ugarte, Pello Bereziartua and Ana Domínguez (again!) allowed testing and validating the application with their students. This work would not have been possible without the input from the teachers, so I am especially grateful for their willingness to help.

The ARETE consortium, after a rough start, was invaluable in providing many interesting discussions and even better social events. The project gave me the opportunity to start working in the field of Augmented Reality applied to education, to learn many new things and to meet amazing professionals.

Many colleagues at NTT DATA, especially Julián González and Diego Rodríguez, gave me their feedback for all the data analysis tasks and they were willing to help whenever I pestered them.

Last but not least, I want to thank Itxaso. She helped and encouraged me throughout the ups and downs of this journey, and she offered her unwavering support during the many evenings and weekends I spent working on my Ph.D. I would not have been able to accomplish this feat without her by my side.

\begin{flushright}

\textit{Thank you all.}

Stefano Masneri

% Moth and year
\monthname \ \the\year
%\today


% Signature figure

\begin{figure}[htbp!]
\hfill\begin{minipage}{.5\textwidth}\centering
\includegraphics[width=0.6\textwidth]{signature.png}
\end{minipage}
\end{figure}
%



\end{flushright}



%Closing of the acknowledgements
%Sort version
%\end{acknowledgements}
% Long version
\end{acknowledgementslong}

% ------------------------------------------------------------------------