\section{Overview}\label{sec:overview}

\gls{ar} is a technology which overlays virtual 3D objects or other content in the real world, with the aim of providing a sense of Mixed Reality \citep{azuma1997survey}. Another widely used definition is from \cite{milgram1994taxonomy} where, in a continuum ranging from a purely virtual environment to a completely real one, AR is positioned close to the real environment and the users perceive the real world with an additional layer of virtuality. This technology was originally used as an aid tool for assembly workers at Boeing by showing them virtual labels through the use of custom made glasses \citep{caudell1992augmented}. Since then, AR has quickly attracted research and industry attention in many different areas such as gaming and entertainment \citep{das2017augmented}, cultural heritage \citep{vlachos2022review}, customer engagement \citep{mclean2019shopping}, manufacturing \citep{ong2008augmented} or education \citep{garzon2019systematic}. 

Despite the appeal and clear advantages provided by AR, the widespread adoption of devices that support the technology, and the availability of software libraries such as ARKit\footnote{\url{https://developers.google.com/ar/}} or ARCore\footnote{\url{https://developer.apple.com/documentation/arkit/}} which simplify the development process; \gls{ar} adoption rate is still low \citep{jalo2022extended}. In most cases, AR applications must be created ad-hoc and cannot be used in different settings. It is not uncommon that technologies considered mature do not achieve widespread usage or, once commercialised, fail to attract interest for the public. One example is that of 3D TV, which suffered from the lack of content and has disappeared from the market. Other technologies, such as artificial neural networks, where initially deployed in commercial application more than 30 years ago \citep{lecun1989handwritten} but only in the last decade have reached ubiquitous diffusion. It happened thanks to improvements in hardware and software as well as the availability of massive amount of data. This led to the possibility of training models which surpassed the state of the art.

\gls{ar} is currently at a turning point: the technology has already reached maturity and is actively being promoted both by hardware producers (Apple is the latest big player who released its AR headset\footnote{\url{https://www.apple.com/apple-vision-pro/}}) and software companies. This seems to suggest that the low adoption of AR is not due to a lack of software solutions enabling the creation of apps, nor to the availability of hardware devices supporting such solutions. The lack of widely accepted standards for the technology and the difficulties of creating portable solutions may be factors affecting the diffusion and acceptance of AR for the general public, but it is of interest to investigate if there are deeper reasons behind the limited availability of AR applications, especially for education \citep{doi/10.2759/121671}.

Unfortunately, AR has not yet seen widespread usage in education, and most studies evaluating AR in schools so far are based on small experiments. Despite this, AR applications have found a valuable role in training and education: several companies offer educational AR apps and many scientific publications have shown that AR can enhance and improve the learning experience \citep{akccayir2017advantages, khan2019impact, garzon2019systematic, GARZON2019244, GARZON2020100334, buchner2022impact, christopoulos2022effects, SAHIN2020103710, Kristoffer2021, huang2016animating, THEODOROPOULOS2021100335}. The ability to attach virtual content to any physical surface (through the usage of markers, by performing plane detection or using geographical information) makes AR applications a valuable tool for training and education. In the context of this research work the focus is on the area of education, specifically on the use of AR in schools, in particular for primary and secondary education.

The main reasons behind the lack of real integration of AR in education, described in more detail in Chapter \ref{chap:arch}, can be summarised as the difficulty developers and educators have in creating content that can be used by every student and that integrates well with the existing school curricula. The majority of AR applications available for education provide single-user experiences and they are more apt to be consumed at home rather than at school. In addition, since most of the existing AR applications are not multi-user, they do not allow students to cooperate. Cooperative learning, defined as the instructional use of small groups to promote students working together to maximise their own and each other's learning \citep{johnson1991cooperation}, has long been used as an educational approach to improve students' learning and performance \citep{Johnson19, kuh2011piecing}.

AR applications for education, and \gls{tel} in general, can also collect data related to usage statistics and students performance. The data should be leveraged by the teachers (through dashboards and visualization, for example) to easily assess the work of the students and their progress. In most cases, though, the data is not accessible. Additionally, the data collected could be used to train \gls{ai} models that could help predict which students are at risk of failing, or cluster them into different groups.

Another issue limiting the adoption of AR in schools is the integration of AR applications within existing school programs. Existing applications cannot be easily adapted to specific school curricula, and the data generated inside the apps (\textit{e.g.}, test results or lesson progress) are not automatically added to a \gls{lms}, thus creating additional workload for teachers. Educational AR applications should ideally use open standard for data collection and storage, to simplify their integration into the school LMS.

This research work therefore aims to investigate how to promote the usage of AR in the classroom through the creation of interactive, multi-user and collaborative mechanisms for AR solutions. This is achieved by closely collaborating with teachers and schools associations, that provided the background information about what is required to successfully adopt this technology in educational environment. The proposed solution should offer a software framework, and it must also consider various factors to encourage the development of an ecosystem. This ecosystem should simplify content personalization and facilitate integration with other applications used in education, allowing adoption regardless of the available hardware in schools.

\section{Motivation}\label{sec:motivation}
The work presented in this dissertation has been performed in the context of the project ARETE\footnote{\url{areteproject.eu}} \citep{masneri2020work}, which aimed to develop, integrate and disseminate interactive technology via AR tools. The project involved academic partners (universities and technological research institutions), \glspl{sme} developing AR solutions for schools and children with learning disabilities, psychologists, educators and institutions connecting schools across all Europe. The consortium thus supported fast dissemination of AR content to a wide audience and allowed the participants to collect feedback from teachers, students and industrial partners. The overall scope of the ARETE project is defined around four main pillars:

\begin{itemize}
    \item Develop and evaluate the effectiveness of an interactive AR content toolkit.
    \item Pilot and evaluate the effectiveness of AR interactive technologies.
    \item Apply human-centred interaction design for ARETE ecosystem.
    \item Communicate, disseminate and exploit the project result.
\end{itemize}

%In particular, the software developed during the project was tested by almost 2000 students and teachers across four different pilot studies, that provided invaluable feedback through standardised qualitative and quantitative studies.

The author of this thesis contributed to this project while working at Vicomtech, whose role in the consortium was to contribute in the definition of the system, provide a set of libraries that can simplify and streamline the development of AR applications with a particular focus on interactivity and ease of use, and enable the collection of data that allows statistical analysis of how the applications developed were used during each pilot. In particular, Vicomtech led the development of \ork{}, a software library that offers multi-user support across different platforms, which in turn enables user to share their current state and send messages and updates with the low latency required in AR environments.

In addition to working with Vicomtech, this Ph.D. work was possible thanks to the affiliation with the GaLan group of the University of the Basque Country, UPV/EHU, a research group integrated into the consolidated ADIAN group. The group works in the field of educational informatics, and its objective is to enhance learning support systems from the perspectives of both teachers and students. For teachers, the group develops tools to aid and simplify course design processes, as well as tools for analyzing the learning process. For students, the group creates tools equipped with adaptation and visualization mechanisms to assist them in learning, reflecting, understanding, and improving their learning processes. The development of such tools involves various technologies derived from the fields of Artificial Intelligence and Information Systems, in addition to agile methodologies and experimental testing and evaluation mechanisms.

Finally, Ikastolen Elkartea and three educational institutions in San Sebastian (IES Xabier Zubiri Manteo, Salesianos Donostia Basque School and University of Deusto) also collaborated in this work. Ikastolen Elkartea, one of the primary and secondary schools association in the Basque Country, helped by organising a series of interviews with teachers as well as with the diffusion of a survey. This allowed the collection of the information that drove the development of the Ph.D. work. The educational institutions participated in the evaluation of the proposed solution. The help of the teachers and of the school association has been invaluable as it allowed bridging the gap between the developers and educators community. They helped in the definition of the architecture described in Chapter \ref{chap:arch} as well as in the evaluation of the application presented in Chapter \ref{chap:eval}.


\section{Objectives}\label{sec:objectives}
The main objective of this work is to explore and propose innovative solutions which use AR in the educational sector, with a strong focus on collaborative, multi-user interactions. This can in turn provide researchers, educators and software developers with tools that simplify the creation and adoption of AR solutions in the classroom, without the need to modify existing school curricula or requiring specific hardware or software setups. Furthermore, the main objective can be decomposed into four more specific objectives:

\begin{itemize}
    \item Objective 1: Identify the main causes behind the lack of adoption of AR solutions in schools, through the analysis of current research and by interviewing teachers who are familiar with information technology.
    \item Objective 2: Develop a software library which enables the creation of multi-user AR applications, where the users can share their experiences across different hardware \textendash{} PCs, laptops, smartphones or \glspl{hmd} \textendash{} and software \textendash{} web browsers, iOS, Unity or Android apps \textendash{}.
    \item Objective 3: Define a modular architecture that fulfills the requirements and design objectives for AR in education, as identified by working with primary and secondary school teachers as well as researchers with education and computer science background.
    \item Objective 4: Validate such architecture by creating a collaborative multi-platform AR application and testing it in a real world scenario, to demonstrate that it fulfills all the design objectives.
\end{itemize}

\section{Hypothesis}\label{sec:hypothesis}
The working hypothesis is constructed as a statement of the following expectations:
\begin{enumerate}
    \item It is possible to enable low-latency communication and state sharing functionalities to AR-based applications across a variety of devices and operating systems.
    \item AR applications supporting all the identified capabilities should be developed just once and compiled for all the required platforms.
    \item AR applications may include data collection capabilities that can be used to perform advanced visualization or for training machine learning models, as well as being integrated into existing LMSs.
    \item Existing AR standards can be leveraged to enhance portability and enable easy customisation of the applications.
\end{enumerate}

These mentioned expectations involve different stakeholders: 
\begin{enumerate}
    \item \textbf{Students}: They are the end users and the ones who will spend the most time using the applications. In an educational context, it is paramount to provide them with the capabilities to collaborate and interact with their peers or their teachers, either in a face-to-face or remote scenario.
    \item \textbf{Teachers}: Teachers have the role of introducing students to the usage of AR and, in order to use AR beyond short-lived experiments, they need to be able to incorporate it into other existing \gls{tel} systems used at school, as well as keep track of the progress of each student and identify potential learning problems.
    \item \textbf{Software developers}: The creation of AR applications is often a very time-consuming task and usually requires developers knowledge of different fields such as 3D geometry, computer vision and \gls{ui} design and \gls{ux} requirements in 3D environments. Simplifying and streamlining the development process is extremely important in every project involving AR content.
    \item \textbf{Researchers}: Since there is limited data available about the analysis and effects of AR applications in education with large amount of students, it is important to provide researchers with all the data collected during the study, as well as open source all the software developed to enable replication and the possibility to extend the current work.
\end{enumerate}

\section{Methodology}\label{sec:methodology}
The research performed in this Ph.D. has followed a methodology consisting of two converging lines, whose work culminated in the definition of the \arch{} architecture and the evaluation in schools of \appname{}, the application used to validated the design objectives defined.

The first line of work relates to the understanding of the problem domain. This involved an in-depth analysis of the literature and, more importantly, discussing with teachers the limitations of current AR solutions and what was needed to encourage its use in their schools. The interactions with teachers were performed several times across the duration of the Ph.D. work. The initial contact was with Ikastolen Elkartea, a Basque association of primary and secondary school teachers. Through the association it was possible to set-up interviews with teachers. The association also helped in the preparation and diffusion of a questionnaire that was filled by more than 40 teachers. Their answers helped greatly in the definition of the requirements and design objectives of \arch.

The other line of work is related to the software developed throughout the Ph.D. work. Since the focus was on research rather than in the creation of a product, the methodology followed in this case was that of fast prototyping development. Each new iteration of the software produced was tested via the creation of simple prototypes, which could be more easily validated and could provide early feedback about strengths and weaknesses of the solution proposed. This approach allowed first the creation of \textit{\ork{}}, a JavaScript library for collaborative web-based applications, and then to extend it to other platforms, improve its performance and use it in AR scenarios.

Finally, the two work lines converged once the development of \appname{} was completed. The application was evaluated in three different schools, thanks to the participation of both teachers and students. Each student filled a questionnaire after testing the application, while the teachers provided feedback through post-study interviews.

\section{Document structure}
This dissertation is structured as follows. Part I is composed by this chapter and presents an introduction to the research scope, focusing on the motivation for the research, the main objectives, the hypothesis, the methodology and the main contributions of the Ph.D. work.

Part II, composed of Chapter \ref{chap:sota}, describes the literature related to AR applications used in education, with a special focus on publications describing interactive, multi-user and collaborative applications.

In Part III the research results are described in two main chapters:
\begin{itemize}
    \item Chapter \ref{chap:arch} describes \ork{}, the library that allows to easily implement multiplatform and multi-user capabilities in AR applications. It also includes the  definition of \arch{}, the architecture that enables the creation of collaborative AR applications and the usage of the collected information for data analysis and predictive modelling.
    \item Chapter \ref{chap:eval} describes the application implemented using the aforementioned architecture, and its evaluation and validation in three educational institutions. The evaluation was done by analysing the data collected from the app in the form of xAPI statements, as well as through user questionnaires filled by the students and post-intervention interviews with the teachers.
\end{itemize}

Part IV contains Chapter \ref{chap:conclusions}, which describes the main conclusions of the research, the contributions of this work and its related publications, as well as a discussion about future work.

Finally, Part \ref{sec:bibliography} contains the bibliography while Part \ref{sec:appendix} provides additional information in the form of an Appendix.