%Introduction
%\begin{savequote}[50mm]
%If our brains were simple enough for us to understand them, we'd be so simple that we couldn't.
%\qauthor{Ian Stewart }%The Collapse of Chaos: Discovering Simplicity in a Complex World
%\end{savequote}
%[Technologies for interoperable multi-device services]
\chapter{Network-aware content encoding}
\chaptermark{Network-aware content encoding}
\label{chap:network-aware}

\section{Context}

The original uncompressed video content undergoes two main operations before being delivered to the user: compression through one of the widely known video codecs, e.g., H.264 or HEVC, and packaging in a media container format, e.g., MPEG-4 Part 14 (commonly called MP4). Encoding and packaging the content influence user's QoE, which plays a significant role when dealing with media services. Thus, optimizing video encoding and packaging strategy contributes to increase the user's satisfaction and to retain the user from leaving the media service. Considering network information and application context (VOD or real-time communications) may lead to a better selection of video encoding bitrate and streaming format/protocol. In this sense, this thesis investigated the possibility of considering such information by designing and implementing two different solutions that exploit it.

MPEG-DASH natively allows encoding bitrate selection at the client side which enables to mitigate network performance fluctuations. This format also fits with the Video on Demand (VOD) scenario where the latency between content packaging and playback is not an issue. On the contrary, it is not suitable when latency constraints come into play. Live streaming applications, such as video surveillance and video conference, cannot work with typical operational ranges meaning tens of seconds of delay of MPEG-DASH.

In Section \ref{chap:BMSB2020}, an Adaptive Rate Control on top of SRT protocol is developed to demonstrate the applicability of network information at the origin server. Differently from MPEG-DASH, SRT protocol is meant for guaranteeing low latency required for Live streaming, but it does not provide the capability to adapt the encoding bitrate. The implemented Adaptive Rate Control-enabled SRT server periodically changes the video resolution and encoding bitrate to adapt live streams accordingly to the information concerning the network throughput and reported by the connected clients. When network throughput decreases, the resolution and encoding bitrate are decreased to prioritize the playback smoothness over video quality. On the contrary, if the throughput increase, encoding bitrate and resolution are also increased. This solution does not need any additional communication, as SRT protocol already provide feedback mechanisms for reporting network status. Then, this paper proposes a real implementation of an Adaptive Rate Control for SRT streams by including the following relevant contributions:
\begin{itemize}
	\item A server-side Adaptive Rate Control implementation on top of open-source framework for SRT streaming applications. This Adaptive Rate Control exploits the network reports employed by SRT protocol to enable the adaptation of the resolution and encoding bitrate of the content.
	\item A coordinated delivery of the stream as the encoding bitrate is chosen by the origin server at once for all the connected media players. It differs from MPEG-DASH, where each client autonomously choses the representation bitrate.
	\item Evaluation of the effects on user's Quality of Experience (QoE) when compared the proposed solution to a legacy one. In both cases, the player does not need any modification as a legacy SRT client can decode the Adaptive Rate Control-enabled stream.
\end{itemize}
Compared to a legacy SRT solution, the results show that the Adaptive Rate Control-enabled SRT delivery experiences fewer freeze events by enabling switching operations to lower representation bitrates. It reduces the average representation bitrate to prioritize playback smoothness. Moreover, in terms of initial delay, there is not a noticeable difference with legacy SRT since the Adaptive Rate Control does not introduce any delay while starting the streaming session.

Section \ref{chap:BMSB2019} presents a study of LL CMAF to deliver Live Streaming which is carried out to evaluate the trade-off between latency and QoE. CMAF is a technological solution which has two major benefits. First, it pushes MPEG-4 Part 14, usually referred as MP4, as a common file format for different streaming technologies, such as MPEG-DASH or HLS. This feature makes media storage more efficient as different manifests (MPD for MPEG-DASH and M3U8 for HLS) may index the same media segments. Therefore, even if the players download different manifests depending on their supported streaming technologies, they download and play the same media segments. Thus, the remote server (origin server or CDN) needs lower storage capacity. Secondly, it defines a low latency mode, also called chunked mode, named LL CMAF or Chunked CMAF, which enables latency enhancement of the stream, reducing the time elapsed between media packaging and its playback. A typical MPEG-DASH segment contains a single MP4 fragment. On the contrary, LL CMAF enables a single segment to contain multiple fragments. A MP4 fragment is the minimum amount of data required by the player to start decoding the stream. Therefore, the shorter fragment duration allows a promptly playback start, removing the limitation to fully download the entire segment, which usually lasts some seconds. This paper includes:
\begin{itemize}
	\item A server-client solution delivering LL CMAF streams on top of open-source framework. 
	\item A comparison with a legacy MPEG-DASH stream having segments of 2 seconds duration to underline the limitations of the setup widely employed for live/low latency video streaming.
	\item The evaluation of the effects on user's QoE while varying the fragment duration and the resulting latency. The employed fragment durations are 33 ms, 100 ms and 167 ms that correspond to fragments containing a Group of pictures (GOP) with 1, 3 or 5 frames for a video with a nominal framerate of 30 frames per second, respectively.
\end{itemize}
The results show that media players gain lower latency in any of the LL CMAF configurations with respects to legacy MPEG-DASH setup. However, when using an aggressive configuration with a small GOP size and fragment duration, the playback has lower protection against freezes which reduce the QoE. To balance the latency and QoE trade-off, a more conservative configuration of LL CMAF is suggested. 
