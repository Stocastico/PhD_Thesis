%Introduction
%\begin{savequote}[50mm]
%If our brains were simple enough for us to understand them, we'd be so simple that we couldn't.
%\qauthor{Ian Stewart }%The Collapse of Chaos: Discovering Simplicity in a Complex World
%\end{savequote}
%[Two-way complementarity of computing capabilities]
%\chapter{Two-way complementarity of computing capabilities in multi-device media services}
\chapter{Network performance forecasts for content delivery}
\chaptermark{Network performance forecasts for content delivery}
\label{chap:predictive}

\section{Context}

In the video streaming context, caching is a fundamental mechanism that aims to prevent negative effects on the QoS/QoE caused by network impairments. A CDN is a widely employed solution to cache and deliver video streams. Furthermore, it is becoming usual for CPs to employ alternative CDNs from different vendors or geographic locations to provide a more reliable service. However, the typical multi-CDN strategy is limited to selecting the CDN to be used at the start of the media session, maintaining it throughout the content playback. Then, the selected CDN is kept along all the streaming session. Moving to more dynamic solutions, that enable to switch between different CDNs when the streaming sessions are ongoing, opens lots of possibilities for optimization. Moreover, employing times series analysis to forecast network performance enables to perform proactive CDN selection and to consider the trade-off between performance (QoS) and costs (Operational Expenditure or OPEX).

Section \ref{chap:IEEETBC2020} proposes to optimize the employed CDN resources by reducing their usage to the effectively necessary moments, when delivering MPEG-DASH streams. The objective is to avoid over-provisioning of CDN resources, as it affects CP's OPEX. The proposed solution, called intelligent network flow (INFLOW), consists in a multi-CDN strategy designed to optimize CDNs utilization and reduce the resultant business costs for it. It exploits periodical MPEG-DASH media presentation description (MPD) updates to apply dynamic switching among the available CDNs at the players in a standard compliant manner. The MPD with the appropriate CDN endpoint is served by the INFLOW Media Server, which works jointly with the INFLOW Forecast Service. The INFLOW Forecast Service provides network metrics predictions based on a Long Short-Term Memory (LSTM) network, a kind of Recurrent Neural Network (RNN), when fed with the historical values of network metrics. The integration of the Forecast Server into the delivery chain allows the Media Server to serve an MPD containing the \textit{BaseURL} of the CDN, which matches target QoS and CP's business requirements. Thus, INFLOW allows for proactive and cost-effective video streaming delivery. This paper comprises the following relevant contributions:
\begin{itemize}
	\item Exploitation of network performance metrics and MPD information to apply common decisions to ongoing streaming sessions. Captured network metrics are employed to forecast CDN serving capacity (throughput) and then to select a CDN only if it would ensure the viability to serve the content at a representation bitrate from the available ones in the MPD that matches with the target minimum QoS.
	\item A dynamic approach switching from a CDN server to another depending on the performed predictions at any time. Thus, in contrast to current solutions, a streaming session is not served from a single CDN provider.
	\item Practical application of a forecast model. The literature proposing a forecast model for QoS network metrics is usually limited to theoretical analysis and simulations where the predictions are not turned into video streaming actions. On the contrary, in this work the predictions are effectively employed to switch the players among the available CDN servers, then proactively acting on the delivery.
	\item Business constraints are considered for the CDN selection. Metrics for both the OPEX and the QoS have been considered in the algorithm which selects the ideal CDN to be employed. Thus, this sophisticated approach favors the dynamic utilization of a CDN marketplace to deal with cost-effective trade-offs. The efficient utilization results in OPEX reduction, while keeping the QoS, which is a major concern for practical deployments in real-world streaming services.
	\item The evaluation includes a comparison with other CDN selection strategies in terms of QoS metrics and business cost.
\end{itemize}
The proposed solution has been implemented and validated in a distributed and heterogeneous testbed employing real network nodes and including both wired and wireless nodes. The wireless nodes were connected through a real Long-Term Evolution (LTE) network deployed with Software Defined Radio (SDR) equipment and OpenAirInterface (OAI) open-source software. The traffic demand on video players was generated according to a probability distribution widely employed in the literature. The results highlight the advantages of INFLOW for reducing the overall usage time of the available CDNs, while guaranteeing a minimum level of network bandwidth to every player.