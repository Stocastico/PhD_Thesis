%Introduction
%\begin{savequote}[50mm]
%If our brains were simple enough for us to understand them, we'd be so simple that we couldn't.
%\qauthor{Ian Stewart }%The Collapse of Chaos: Discovering Simplicity in a Complex World
%\end{savequote}

\section{Conclusions}

This Ph.D. thesis has identified different research forks to address the main objective to improve QoS and QoE of media streaming, while reducing CP's business costs. In this sense, the exploitation of the information, acquired both from media content and from network analysis, has been decisive to increase the performance of video streaming services. The contributions have approached different stages and/or network functions involved in the video streaming workflow.

Concerning the first contribution area, \textit{Network-aware video encoding}, strategies to encode and package the video content have been studied. Two different solutions have been designed to take into account network status when preparing the video content for streaming.
The first solution enables to tune the video encoder at the SRT server accordingly to the network status. When the network throughput cannot cope with the demanded rate of the video content, the encoding bitrate and resolution are decreased to prioritize the playback smoothness over video quality. In the same way, when the throughput increase, encoding bitrate and resolution are also increased. The implemented SRT server works with compliant SRT clients without any modifications.
The second solution evaluates the use of LL CMAF to reduce latency when delivering MPEG-DASH streams. Effectively, media players experience lower latency compared to a legacy MPEG-DASH solution. The latency and user's QoE trade-off is also evaluated by varying the encoding and packaging configurations, i.e., changing the GOP and fragment duration. When using an aggressive configuration with a small GOP and fragment duration, the playback is frequently affected by freezes which damage the QoE. Then, a more conservative configuration of LL CMAF is suggested to keep QoE scores.

The research on the second contribution area, \textit{Network performance forecast for video delivery}, has investigated the use of ML algorithms to analyze network metrics and forecast performance. In a multi-CDN context, being able to forecast CDN performance means enabling a better CDN selection for the CP and reducing business cost for CDN usage.
A solution that employs an LSTM model has been proposed and trained to provide CDN performance forecasts based on time series analysis of the collected network metrics. The integration of the LSTM model into the delivery chain and the exploitation of the information included in the MPEG-DASH MPD allowed the media server to take actions that enforce the delivery. The media server was able to modify the MPD to force the players to download the media segments from the more appropriate CDN that matches target QoS and CP's business requirements.

Finally, the research on the third contribution area, \textit{MEC-enabled video delivery}, has led to de implementation of services to be employed on top of the novel 5G MEC architecture. The first solution consists in a MEC proxy that estimates the users' QoE according to ITU-T P.1203. QoE scores are derived by inferring monitored QoS metrics and the information acquired by parsing the MPEG-DASH MPD. It works independently of the video servers and players, as it does not need an explicit out of band messaging. The awareness of QoE values is an important enabler for advanced solutions to enforce the QoE at the MEC platform.
In the second solution, MEC location is exploited to provide a service to enable a MEC-empowered delivery having two main advantages over legacy server-client communication. First, it proactively caches MPEG-DASH segments at network edge to reduce cloud CDN usage. Second, it shields from identified or predicted CDN malfunctions by switching the download of segments to an alternative CDN in order to ensure QoE rates.
Thus, the implemented MEC service allows to keep the QoE scores by switching to healthy CDNs or even improve them by proactively caching the content at the edge.

In a nutshell, this research work provides progress beyond the state-of-the-art for video streaming. Architectures, systems and algorithms have been proposed to advance in three contribution areas. The feasibility of all the contributions has been demonstrated through the implementation of the proposed solutions and their deployment in operational and realistic setups. The results obtained have been also compared with legacy solutions to provide evidence of the improvements introduced in video streaming.


\section{Future work}

During the development of research activities, literature review, design and implementation of solutions, and analysis of results, several future research lines have been identified to complement or extend the research presented in the contribution areas of this thesis.

Regarding the first contribution, the identified future works are:
\begin{itemize}
	\item \textit{Business cost for encoding:} once the video codec has been chosen, the encoding operations may have operational costs that vary depending on the encoder choice, i.e., open-source or commercial, and where the encoder runs, i.e., cloud or on-premise encoding. Cloud encoding prices are established by cloud providers, while on-premise encoding depends on the hardware selection and maintenance. On the other side, on-premise encoding allows to have more control on the processed content compared to cloud encoding. In this context, the efficiency of the encoding operations could be furtherly increased by including considerations on business cost.
	\item \textit{End-to-end latency:} while for VOD streams, HAS solutions, such as MPEG-DASH and HLS, are de facto standard, Live streaming still resists from a widely adoption of such solutions. This is still valid when enhancing HAS with LL CMAF. The reason is quite obvious as HAS cannot still compare in terms of latency with protocols originally designed for low latency applications. In this sense, WebRTC has raised in the last few years and is proliferating thanks also to the COVID-19 pandemic. As a drawback, WebRTC is not simple to scale since it employs specific signaling protocols, such as STUN and TURN, which add bootstrapping signaling and overheads when compared to HAS solutions. On one side, the future research will investigate how to improve LL CMAF and push its adoption. On the other side, it will try to increase WebRTC scalability.
\end{itemize}

Then, concerning the contributions of the second area, the future research activities should address:
\begin{itemize}
	\item \textit{Advanced network metrics:} this research has focused on employing network layer measurements (bandwidth and latency) to train a ML model and exploit the predictions jointly with application layer information (MPEG-DASH MPD). The model can be furtherly improved by collecting more complex metrics, including data link layer information, such as transmitted packets or packet losses. The higher the number of metrics processed, the more accurate the ML model would be.	
	% Increasing the number of metrics could increase the accuracy of the ML model.
	\item \textit{Complex time series models:} research work in literature concludes that there is not an optimal time series model, as the selection depends on the particular application or physical network. New studies are investigating the possibility of combining different models at the same time. The idea is to exploit advantages of each model to provide better forecast results.
	\item \textit{SDN integration:} the investigated solution takes actions to optimize the streaming process when the network assets' capabilities vary. Such optimization is limited to act on top of the network (changes are only operated at the media server), as no changes are applied at network layer. The traffic between server and clients are still transmitted on a best-effort basis. Here, it is interesting to move the optimization also to the network layer. It means enabling the direct management of the network assets and not just limiting monitoring them. In this context, the integration of SDN represents a further step. Employing a SDN controller to guarantee the necessary network layer capabilities between CDN and player and designing its cooperation with network functions (origin and media servers, players, etc.) create a more reliable and efficient end-to-end streaming system.
\end{itemize}

Finally, the future lines related to the third contribution are:
\begin{itemize}
	\item \textit{RNI standardization and processing algorithms:} the API to access RAN information or RNI has been recently standardized and its development is still ongoing. When RNI Service (RNIS) implementations will be available, services running at the MEC host can be further optimized and embed more complex and precise algorithms. Improved algorithms will exploit RNI in order to adjust the operations and the performance of the overall system.
	\item \textit{Business model and hardware acceleration:} MEC is still missing a business model equivalent to the one applicable in cloud computing infrastructures. However, unlike cloud computing, the decentralized location and utilization of shared resources between services makes the cost model more complex. Resource accounting and monitoring have to be determined in order to create a complete business model. The debate on the business model is even more intricate if hardware-acceleration assets, such as GPUs, are considered. Integrating GPU clearly provides capabilities to accomplish critical tasks where general-purpose hardware (CPU) has limitations. Once the use of GPU and the corresponding business model is clear, the debate on how to optimize resource and business cost trade-off could be raised.
	\item \textit{Mobility:} the explosion in availability and type of mobile devices (e.g., smartphone and tablets) involves an increasing number of UEs to be served. Thus, mobility remains a major concern. The same way the connectivity is guaranteed when moving from a cell to another in a cellular network, migration support for MEC services is also required. Consequently, the investigation on a multi-MEC cooperation should be addressed in order to guarantee seamless migration of sessions across MEC hosts.
\end{itemize}