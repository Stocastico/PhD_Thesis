%Introduction
%\begin{savequote}[50mm]
%If our brains were simple enough for us to understand them, we'd be so simple that we couldn't.
%\qauthor{Ian Stewart }%The Collapse of Chaos: Discovering Simplicity in a Complex World
%\end{savequote}
\chapter{Other publications}
\label{chap:otherpublications}

Apart from the publications directly related with this thesis, the following list shows other publications carried out by the author of this work.

\section{International journals}

\subsection{Journal paper OJ1}
\label{chap:TBC2018}
\textbf{Title:} Network Resource Allocation System for QoE-Aware Delivery of Media Services in 5G Networks \\
\textbf{Authors:} \'Angel Mart\'in, Jon Egaña, Julián Flórez, Jon Montalbán, Igor G. Olaizola, Marco Quartulli, Roberto Viola and Mikel Zorrilla \\
\textbf{Journal:} IEEE Transactions on Broadcasting \\
\textbf{Pages:} 561-574 \\
\textbf{Publisher:} IEEE \\
\textbf{Year:} 2018 \\
\textbf{DOI:} \url{10.1109/TBC.2018.2828608} \\
\textbf{Abstract:} \textit{The explosion in the variety and volume of video services makes bandwidth and latency performance of networks more critical to the user experience. The media industry's response, HTTP-based Adaptive Streaming technology, offers media players the possibility to dynamically select the most appropriate bitrate according to the connectivity performance. Moving forward, the telecom industry's move is 5G. 5G aims efficiency by dynamic network optimization to make maximum use of the resources to get as high capacity and Quality of Service (QoS) as possible. These networks will be based on software defined networking (SDN) and network function virtualization (NFV) techniques, enabling self-management functions. Here, machine learning is a key technology to reach this 5G vision. On top of machine learning, SDN and NFV, this paper provides a network resource allocator system as the main contribution which enables autonomous network management aware of quality of experience (QoE). This system predicts demand to foresee the amount of network resources to be allocated and the topology setup required to cope with the traffic demand. Furthermore, the system dynamically provisions the network topology in a proactive way, while keeping the network operation within QoS ranges. To this end, the system processes signals from multiple network nodes and end-to-end QoS and QoE metrics. This paper evaluates the system for live and on-demand dynamic adaptive streaming over HTTP and high efficiency video coding services. From the experiment results, it is concluded that the system is able to scale the network topology and to address the level of resource efficiency, required by media streaming services.} \\
\hrulefill

\subsection{Journal paper OJ2}
\label{chap:TBC2019}
\textbf{Title:} MEC for Fair, Reliable and Efficient Media Streaming in Mobile Networks \\
\textbf{Authors:} \'Angel Mart\'in, Roberto Viola, Mikel Zorrilla, Julián Flórez, Pablo Angueira and Jon Montalbán \\
\textbf{Journal:} IEEE Transactions on Broadcasting \\
\textbf{Pages:} 264-278 \\
\textbf{Publisher:} IEEE \\
\textbf{Year:} 2019 \\
\textbf{DOI:} \url{10.1109/TBC.2019.2954097} \\
\textbf{Abstract:} \textit{Beyond the advanced radio capabilities, 5G means a digital transformation, catalyzed by cloud technologies, making the networks agile and broader. However, high and quick dynamics in dense client cells consuming live broadcast contents can cause Quality of Experience (QoE) degradations. Here, inaccurate bandwidth assessment of media players drives to buffering times along with quality fluctuations. Moreover, massive recurrent requests can negatively impact on Content Delivery Network (CDN) performance. Complemented by capillarity and zero-latency features of multi-access edge computing (MEC) systems, 5G infrastructures will expand media services to take QoE to a new level. This paper investigates QoE gains of an MEC enabled infrastructure. The proposed MEC system applies three video delivery mechanisms. First, it enforces the QoE in a congested cell. Second, it shields from CDN degradation for a reliable content distribution. Third, it enhances network core and backhaul efficiency saving CDN traffic. Furthermore, our solution is deployed and tested on a LTE infrastructure. Results for live streams show that the MEC system makes the media players tend to a common and high quality bitrate, and it is able to quickly, transparently and coordinately switch to healthy CDN infrastructures and reduce CDN traffic.} \\
\hrulefill

\subsection{Journal paper OJ3}
\label{chap:RTIP2017}
\textbf{Title:} LAMB-DASH: A DASH-HEVC adaptive streaming algorithm in a sharing bandwidth environment for heterogeneous contents and dynamic connections in practice \\
\textbf{Authors:} \'Angel Mart\'in, Roberto Viola, Josu Gorostegui, Mikel Zorrilla, Julian Florez and Jon Montalban \\
\textbf{Journal:} Journal of Real-Time Image Processing \\
\textbf{Pages:} 2159-2171 \\
\textbf{Publisher:} Springer Berlin Heidelberg \\
\textbf{Year:} 2019 \\
\textbf{DOI:} \url{10.1007/s11554-017-0728-x} \\
\textbf{Abstract:} \textit{HTTP Adaptive Streaming (HAS) offers media players the possibility to dynamically select the most appropriate bitrate according to the connectivity performance. A best-effort strategy to take instant decisions could dramatically damage the overall Quality of Experience (QoE) with re-buffering times, and potential image freezes along with quality fluctuations. This is more critical in environments where multiple clients share the available bandwidth. Here, clients compete for the best connectivity. To address this issue, we propose LAMB-DASH, an online algorithm that, based on the historical probability of the playout session, improves the Quality Level (QL) chunk Mean Opinion Score (c-MOS). LAMB-DASH is designed for heterogeneous contents and changeable connectivity performance. It removes the need to access a probability distribution to specific parameters and conditions in advance. This way, LAMB-DASH focuses on the fast response and on the reduced computing overhead to provide a universal bitrate selection criterion. This paper validates the proposed solution in a real environment which considers live and on-demand Dynamic Adaptive Streaming over HTTP (DASH) and High-Efficiency Video Coding (HEVC) services implemented on top of GStreamer clients.} \\
\hrulefill

\section{International conferences}

\subsection{Conference paper OC1}
\label{chap:BMSB2020-1}
\textbf{Title:} Realising a vRAN based FeMBMS Management and Orchestration Framework \\
\textbf{Authors:} Alvaro Gabilondo, Javier Morgade, Roberto Viola, Juan Felipe Mogollón, Mikel Zorrilla, Pablo Angueira and Jon Montalbán \\
\textbf{Conference:} 2020 IEEE International Symposium on Broadband Multimedia Systems and Broadcasting (BMSB) \\
\textbf{Pages:} 1-7 \\
\textbf{Publisher:} IEEE \\
\textbf{Year:} 2020 \\
\textbf{DOI:} \url{10.1109/BMSB49480.2020.9379891} \\
\textbf{Abstract:} \textit{FeMBMS is the first broadcast only profile standardized in 3GPP. Re1-14 enables large scale transmission of multimedia content to mobile portable devices including free to air reception of TV services. While the new specification already meets most of the 5G-Broadcast requirements it is also expected to be further evolved in future 5G/3GPP releases. Moreover, in parallel to 5G standardization, a transition in the Radio Access Network (RAN) infrastructure is also taking place, transition where the virtualization of radio access technologies through the use of commodity processing hardware promises to make an end-to-end cloud based 5G network infrastructure a reality. In this paper we investigate first the potential of vRAN based 5GBroadcast networks. Later, based on OpenAirInterface and the containerization of its components, we introduce the development and analysis of a Kubernetes based FeMBMS end-to-end network architecture. The results address, among others, the potential of vRAN to foster the broadcast industry requirements in a 3GPP ecosystem.} \\
\hrulefill

\subsection{Conference paper OC2}
\label{chap:BMSB2019-1}
\textbf{Title:} L3 and L7-driven Dynamic Throughput Balancing over Cellular Networks \\
\textbf{Authors:} Alvaro Gabilondo, Roberto Viola, \'Angel Mart\'in, Mikel Zorrilla and Jon Montalbán \\
\textbf{Conference:} 2019 IEEE International Symposium on Broadband Multimedia Systems and Broadcasting (BMSB) \\
\textbf{Pages:} 1-6 \\
\textbf{Publisher:} IEEE \\
\textbf{Year:} 2019 \\
\textbf{DOI:} \url{10.1109/BMSB47279.2019.8971949} \\
\textbf{Abstract:} \textit{Broadcast of live sports and events often requires the coverage of a wide area and portable transmission units for the mobile cameras. In this context, the mobile network aspires to be a professional tool companion for media production to boost mobility and alleviate costs, space and specialist maintenance of satellite equipment. Transmission of live high quality captured video and graphic design to a cloud or distant studio production infrastructure requires high uplink data rates. However, steady and reliable communications are challenging for the network in disperse, distant and sparse areas. This context may need bonding multiple cellular links to ensure a sufficient Quality of Service (QoS). Video uplink solutions at different network layers can shield from QoS degradation. Communications industry solution for IP bonding consists on having different Long-Term Evolution (LTE) network interfaces with several Subscriber Identity Module (SIM) cards on the device which transmits the live stream, then having network redundancy. This paper provides an innovative method to dynamically balance the throughput for each concurrently employed network interface in real-time at the live video transmitter. The solution exploits live measurements obtained from the network layer (L3), such as network bandwidth, latency and jitter, which are periodically assessed along the video transmission, and application layer (L7) state, such as the encoding Group Of Pictures (GOP) schema, frame type and framerate, to split the video packets in the different network interfaces. The evaluation of the solution is made for a head-end implementation by sending live video streams and measuring the QoS at the production infrastructure. To conclude the benefits when the solution comes into play, results are compared to a scenario without bonding solutions and another one where balance rates are initially fixed.} \\
\hrulefill

\chapter{Resume}

Roberto Viola received his Computer and Telecommunication Engineering degree in 2014 and an advanced degree in Telecommunication Engineering in 2016 from University of Cassino and Southern Lazio (Italy). Currently, he is Research Associate, as part of Digital Media department of Vicomtech. He is involved in R\&D projects dealing with multimedia services and network infrastructure. At the same time, he is working on his PhD degree on video streaming in 5G networks at the University of the Basque Country (UPV/EHU).
