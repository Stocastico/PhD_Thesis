% !TeX spellcheck = en_GB

% Thesis Abstract -----------------------------------------------------


%\begin{abstractslong}    %uncommenting this line, gives a different abstract heading


\begin{abstracts}        %this creates the heading for the abstract page
%\selectlanguage{english}
\addcontentsline{toc}{chapter}{Abstract}
%Put your abstract or summary here.
The current 5G deployment of 5G networks is bringing new network capabilities, including enhanced mobile broadband, ultra-low latency and massive device connectivity. These enhancements are also pushed by the increasing popularity of wireless and mobile devices. Moreover, softwarization and virtualization technologies, such as Software Defined Network (SDN) and Network Function Virtualization (NFV), are considered as key pillars of 5G, as well as for network generations beyond 5G.
NFV paradigm virtualizes all the data centers that are part of the network infrastructure (Core, Edge and Access Networks) to bring cloud technologies into the network operations,
while SDN technology centralizes network control and manages the forwarding rules between data centers to adapt networking policies to traffic demands.
The combination of them enables to operate and manage network functions, referred as Virtual Network Functions (VNFs), by software running on top of general-purpose hardware. The usage of NFV and SDN can be extended also to the Access Network, where Multi-access Edge Computing (MEC) represents a new architectural paradigm to provide cloud capabilities closer to the clients. MEC allows the deployment of edge services to empower heterogeneous vertical applications.

At the same time, we are witnessing a growth in the usage of video streaming applications, including commonly used services, such as Live Streaming and Video-on-Demand (VOD), and new media applications, such as online gaming and 3D video applications (eXtended Reality, Virtual Reality and Augmented Reality). Moreover, streaming solutions tailored to empower different vertical applications, e.g., Industrial Internet of Things (IIoT), medical imaging and automotive machine-vision, are gaining relevance. These trends in video streaming are shaping network traffic, where 5G networks are expected to cope with the increasing total network traffic, mostly generated by media services.

In this context, technologies included in the 5G ecosystem are considered to be the enablers to overcome the challenges raised by the increasing media content generation and consumption. New network functions should be designed and implemented on top of the 5G infrastructure to support high Quality of Service (QoS) and Quality of Experience (QoE) required by media applications and end users. Moreover, these network functions should exploit metrics coming from the network concerning connectivity performance and player's traffic demand to adapt to changeable network conditions. Enabling dynamic changes during the operation of the streaming system has effects also on Content Provider's business costs, as network functions could be optimized to reduce resource usage to what is actually needed. Then, network resources and their costs can be balanced by defining business rules.

In this Ph.D. thesis, the main objective is to improve video streaming QoS and user's QoE, while reducing CP's business costs, through three different contribution areas. Contributions addressed several stages, such as content encoding and delivery, and network nodes, such as origin server, Content Delivery Network (CDN) and MEC host, involved in video streaming workflow. In any of them, the exploitation of the information, acquired from the analysis of both media content characteristics and network metrics, was decisive for increasing the performance of the streaming system.

First, concerning \textit{Network-aware video encoding}, investigating strategies for encoding and packaging the video content has led to two different implementations that leverage network information when preparing the video content for streaming.
In the first one, on top of the Secure Reliable Transport (SRT) protocol, the encoding and packaging configurations are tuned to keep QoS/QoE rates when network capabilities change, to prioritize playback smoothness over video quality.
In the second one, the use of Low Latency Common Media Application Format (LL CMAF) has been studied to reduce latency when delivering Dynamic Adaptive Streaming over HTTP (MPEG-DASH) streams. The trade-off between latency and user's QoE is demonstrated to be an important factor when selecting the encoding and packaging configurations.

Then, regarding \textit{Network performance forecast for video delivery}, the use of Machine Learning (ML) techniques to analyze network metrics has been investigated. A solution that exploits a Long Short-Term Memory (LSTM) model has been implemented to forecast network performance and enhance the selection of a CDN, when multiple CDN are employed, to deliver MPEG-DASH streams. Being able to forecast CDN performance allows the selection of the CDNs according to defined business rules. Then, a trade-off between QoS and business costs for CDN usage is evidenced.

Finally, this thesis has contributed to \textit{MEC-enabled video delivery} with the implementation of two MEC services to be employed on top of the novel 5G MEC architecture. The first solution assesses the user's QoE according to ITU-T P.1203 since QoE knowledge is an important enabling factor for advanced solutions to enforce QoE on the MEC platform. Thus, this solution infers QoE from QoS information collected from MPEG-DASH Media Presentation Description (MPD) and from network monitoring.
In the second solution, a MEC service is deployed to enforce and boost MPEG-DASH streams. The solution provides two operations. First, it enables a proactive cache of MPEG-DASH segments at the network edge to reduce CDN usage. Second, it shields from CDN malfunction by switching the download of segments to an alternative CDN to ensure QoE rates. This MEC service supports media playback with steady QoE scores by switching from one CDN to another and enhances it by proactively caching the content.


\end{abstracts}

%\begin{resumen}        %this creates the heading for the abstract page
%%\selectlanguage{spanish}
%Añadir aquí un resumen.
%
%
%\end{resumen}


%\begin{laburpena}        %this creates the heading for the abstract page
%\selectlanguage{basque}
% Jarri zure laburpena hemen.
%Laburpena Euskaraz.

%\end{laburpena}

%\end{abstractlongs}


% ---------------------------------------------------------------------- 
