% !TeX spellcheck = en_GB

% Thesis Abstract -----------------------------------------------------


%\begin{abstractslong}    %uncommenting this line, gives a different abstract heading


\begin{abstracts}        %this creates the heading for the abstract page
%\selectlanguage{english}
% Put your abstract or summary here.
The evolution of Internet-connected devices, as well as the changes in the way media is produced, distributed and consumed, have promoted the mobility and ubiquity of broadcast and media services. Consequently, the audiovisual sector has been transformed into a hybrid ecosystem where content is distributed across multiple devices. Moreover, a single user very often consumes content from more than one device at a time. 

This novel context brings highly flexible experiences where the content not only has to be adapted to any target device but also requires an adaptation to multi-device environments, composed of a number of devices that are being used simultaneously by one or multiple end-users sharing an experience remotely.
Accordingly, the user interface becomes a key factor to facilitate understanding the application and to provide an intuitive interaction method across multiple screens.

However, to the best of our knowledge, no existing adaptation models are available to dynamically and seamlessly adapt such a multitude of content to multi-device and multi-user contexts.

To address this gap, this research proposes a methodology which provides as an outcome an adaptation model for the user interface of multi-device media services within the audiovisual and broadcasting field, general enough to be easily adapted to many different use cases and scenarios and ready for any technological update. 

The proposed methodology is the outcome of a extensive research that arose from a deployment of a hybrid Broadcast-Internet multi-device service with broadcasters, which has shown a set of hints to consider. This hints enabled to identify and characterise the user interface elements according to a set of design factors and formalise an adaptation model for hybrid broadcast-Internet multi-device services, which has been implemented and validated in terms of quality, efficiency and universality.

Finally, this research also specifies how the mentioned methodology could be applied to other fields different from the broadcast analysing the challenges, the user needs and the specifications in each field.




\end{abstracts}

\begin{resumen}        %this creates the heading for the abstract page
%\selectlanguage{spanish}

La evolución de los dispositivos conectados a Internet, así como los cambios en la manera en la que se produce, distribuye y consume el contenido multimedia, ha fomentado la movilidad y ubicuidad de los servicios multimedia. En consecuencia, el sector audiovisual se ha transformado en un ecosistema híbrido en el que el contenido se distribuye a través de múltiples dispositivos. Además, cada usuario individual a menudo consume contenido desde más de un dispositivo al mismo tiempo. 

Este nuevo contexto posibilita experiencias altamente flexibles donde el contenido no solo tiene que ser adaptado a cualquier dispositivo objetivo, sino que también requiere una adaptación a entornos multi-dispositivo compuestos por un gran número de dispositivos que se utilizan simultáneamente por uno o múltiples usuarios que comparten la experiencia de manera remota. De este modo, la interfaz de usuario se convierte en un factor clave para facilitar la comprensión de la aplicación y para proveer una interacción intuitiva a través de las múltiples pantallas. 

Sin embargo, a nuestro entender, no existen modelos de adaptación para adaptar tal cantidad de contenido a contextos multi-dispositivo y multi-usuario de una manera dinámica y continua. 

Para abordar esta necesidad, esta investigación propone una metodología que provee como resultado un modelo de adaptación para la interfaz de usuario de servicios multimedia multi-dispositivo en el mundo audiovisual y broadcast, suficientemente general como para adaptarlo facilmente a diferentes casos y escenarios y preparado para cualquier actualización tecnológica.  

La metodología propuesta es el resultado de una extensa investigación que comenzó con el despliegue de un servicio híbrido broadcast-Internet multi-dispositivo con broadcasters, del que se extrajeron una serie de \mbox{lecciones} a tener en cuenta. Estas lecciones permitieron la identificación y caracterización de los elementos más relevantes de la interfaz de usuario de acuerdo a una serie de factores de diseño, así como la formalización de un modelo de adaptación para servicios híbridos broadcast-Internet multi-dispositivo. Este \mbox{modelo} ha sido implementado y validado en términos de calidad, eficiencia y universalidad. 

Por último, esta investigación también especifica como se podría aplicar la metodología mencionada en otros campos distintos del broadcast, \mbox{analizando} los retos, las necesidades de los usuarios y las especificaciones requeridas por cada campo. 


\end{resumen}


%\begin{laburpena}        %this creates the heading for the abstract page
%\selectlanguage{basque}
% Jarri zure laburpena hemen.
%Laburpena Euskaraz.

%\end{laburpena}

%\end{abstractlongs}


% ---------------------------------------------------------------------- 
